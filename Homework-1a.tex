\documentclass[12pt]{article}
\usepackage{listings}
%------------------------------------------------------------------------------------
\begin{document}
\begin{titlepage}
   \begin{center}
       \vspace*{1cm}
       \large
       \textbf{Homework 1: Introduction to Algorithmic Analysis and recurrence}
       \normalsize

       \vspace{0.5cm}

       \textbf{Author: Gabriel Hofer}

       \vspace{0.5cm}

       CSC-372 Analysis of Algorithms

       \vspace{0.5cm}

       Instructor: Dr. R

       \vspace{0.5cm}

       Section 1 DUE: Thursday, Aug 27th, at 7AM  \newline
       Section 2 DUE: Thursday, Sept 3 th, at 7AM  

       \vfill

       Department: Computer Science and Engineering\\
       University: South Dakota School of Mines and Technology\\

   \end{center}
\end{titlepage}
%------------------------------------------------------------------------------------
\newpage
\section{Introductory Information}

\begin{enumerate}

  \item (3 pt) How soon do you need to notify me for a normal extension? 
    36 hours

  \item (3 pt) How many projects will there be? 
    5 projects

  \item (3 pt) How long do you have to notify me for a possible grading error, starting when? the grade
    One week

  \item (3 pt) What is the ONLY option to bring up your grade at the end of the semester? Se
    Take the optional second chance 

  \item (8 pt) When did you attend ZOOM office hours after Aug 19 (this will confirmed later)?
    August 20, 2020  

  \item (3 pt) Should your microphones/video initially be on or off when attending a Zoom recitation/office hours.
    Start with camera and microphone off.

  \item (3 pt) What topic(s) are tentatively planned for Oct. 9?
    F: Closest Pair of points

  \item (3 pt) At minimum view, the entry quiz (competition is not required)

    %  ------------------------End: Graded all or nothing------------------------------------------
  \item 9. (6 pt) What is the run time for the following code. You MUST show your work for any credit
    Run Time = O(x * y * z * n)

\end{enumerate}



    

\section{Coding}

a. (40 pt) CODE the book’s version of insertion sort and merge sort. Mark these with
comments in the format; “GRADING: INSERT” and “GRADING: MERGE”, respectively. This is
so we can check the code quickly. Accuracy checked with correct input\\output.










\end{document}

