\documentclass[12pt, a4paper]{article}
\usepackage{listings}
\usepackage{pdfpages}
\usepackage[legalpaper, margin=1in]{geometry}
\usepackage{amsmath}

%------------------------------------------------------------------------------------
\begin{document}
\begin{titlepage}
   \begin{center}
       \vspace*{1cm}
       \large
       \textbf{Homework 1b: Introduction to Algorithmic Analysis and Recurrence}
       \normalsize

       \vspace{0.5cm}

       \textbf{Author: Gabriel Hofer}

       \vspace{0.5cm}

       CSC-372 Analysis of Algorithms

       \vspace{0.5cm}

       Instructor: Dr. Rebenitsch

       \vspace{0.5cm}

       Section 1 DUE: Thursday, Aug 27th, at 7AM \\ 
       Section 2 DUE: Thursday, Sept 3 th, at 7AM  

       \vfill

       Department of Computer Science and Engineering\\
       South Dakota School of Mines and Technology\\

   \end{center}
\end{titlepage}
%------------------------------------------------------------------------------------
\newpage
\subsection*{Section 2: Recursion Analysis}

\textbf{1) (26 pt) Determine the run time (bit-O) for the following recurrence formula using the tree or substitution method.
You may use the master method only to check your answer. } \\ 

\[
  T(n) =  
  \begin{cases}
    1 & n = 1  \\
    3T(\frac{n}{4}) + n^2 & n > 1  \\
  \end{cases}
\]

Substitution Method: \\ 

1. We guess that the form of the solution is $ T(n) = O(n^2 log(n)) $.

2. We want to show by mathematical induction that $ T(n) \leq d \cdot n^2 log(n) $ for some constant $ d > 0 $. 

\[
  T(n) = 3T \left(\frac{n}{4} \right) + n^2
\]
\begin{center}
  Now substitute for the $T \left(\frac{n}{4} \right)$ term in the above equation:
  \end{center}
\[
  T(n) = 3 \cdot d \left(\frac{n}{4} \right)^2 \cdot log \left(\frac{n}{4} \right) + n^2
\]
\[
  = \frac{3}{16} d \cdot n^2 \cdot log \left(\frac{n}{4} \right) + n^2
\]
\[
  \leq d \cdot n^2 \cdot log(n) + n^2
\]
\begin{center}
  When d is sufficiently large we drop the $ n^2 $ term:
\end{center}
\[
  d \cdot n^2 \cdot log(n)
\]
\\

\textbf{2) (26 pt) Determine the run time (big-O) for the following recurrence formula using the tree or substitution method.
You may use the master method only to check your answer.} \\

\[
  T(n) =  
  \begin{cases}
    1 & n = 1  \\
    2T(\frac{n}{3}) + n^3 & n > 1  \\
  \end{cases}
\]

Substitution Method: \\ 

1. We guess that the form of the solution is $ T(n) = O(n^3 log(n)) $.

2. We want to show by mathematical induction that $ T(n) \leq d \cdot n^3 log(n) $ for some constant $ d > 0 $. 

\[
  T(n) = 2T \left(\frac{n}{3} \right) + n^3
\]
\begin{center}
  Now substitute for the $T \left(\frac{n}{3} \right)$ term in the above equation:
\end{center}
\[
  T(n) = 2 \cdot d \left(\frac{n}{3} \right)^3 \cdot log \left(\frac{n}{3} \right) + n^3
\]
\[
  = \frac{1}{3} d \cdot n^3 \cdot log \left(\frac{n}{3} \right) + n^3
\]
\[
  \leq d \cdot n^3 \cdot log(n) + n^3
\]
\begin{center}
  When d is sufficiently large we drop the $ n^3 $ term:
\end{center}
\[
  d \cdot n^3 \cdot log(n)
\]
\\
\newpage



3) (12 pt) Determine which case of the Master Theorem applies for the following recurrences. Include the values of
a, b, and k (and ideally $b^k$) as proof of your selection. Also, include the final big-theta formula. 
You also have teh option of a recurrence relation that cannot use the master method as described in class, 
in which case, state it "fails". \\

  a. $ T(n) = 2T (\frac{n}{2}) + \sqrt{n} $


\newpage
4)

5)

\textbf{6) (18 pt) Write the resulting recurrence relation (the T(n) piece-wise function) for the
following pseudocode where A is an arrays of integers: } \\

\begin{lstlisting}
  FUNC(A, s, e)
      if s >= e
          print s, e, and all of A[1..n]
          return
      cut1 = (e - s) / 2
      cut2 = (e - s) / 4
      FUNC(A, s, s + cut1)
      FUNC(A, s + cut1 + 1, s + cut1 + cut2)
      FUNC(A, s + cut1 + cut2 + 1, e)
\end{lstlisting} 

Recurrence relation solution: 
\[
  T(n) = T \left( \frac{n}{2} \right) + 2T \left( \frac{n}{4} \right) + n;
\]

Explanation: First, an assumption that we make in our analysis is that
array, A, is passed by reference so that A doesn't have to be copied 
for every function call. 
There are three recursive function calls in FUNC. 
cut1 is half of the distance between the start, $s$, and end, $e$, of the current segment 
of the array. cut2 is one forth of the distance between the start and end
of the current segment of the array.
Each of the three recursive function calls pass arguments which effectivly shorten the range 
between the start and end of the array.
The first function call, FUNC(A, s, s + cut1), essentially passes the first half of the current
segment of A.  
The second and third function calls, FUNC(A, s + cut1 + 1, s + cut1 + cut2) and 
FUNC(A, s + cut1 + cut2 + 1, e), both essentially pass a quarter of the current segment of 
the array.
Thus: 

\[ 
  T(n/2) + T(n/4) + T(n/4) = T(n/2) + 2T(n/4)
\]

Since the whole array is printed if $ s \geq e $ for any given function call, we add $ n $ to the total runtime 
of $T$.



\end{document}







