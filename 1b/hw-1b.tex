\documentclass[12pt, a4paper]{article}
\usepackage{listings}
\usepackage{pdfpages}
\usepackage[legalpaper, margin=1in]{geometry}
\usepackage{amsmath}

%------------------------------------------------------------------------------------
\begin{document}
\begin{titlepage}
   \begin{center}
       \vspace*{1cm}
       \large
       \textbf{Homework 1b: Introduction to Algorithmic Analysis and Recurrence}
       \normalsize

       \vspace{0.5cm}

       \textbf{Author: Gabriel Hofer}

       \vspace{0.5cm}

       CSC-372 Analysis of Algorithms

       \vspace{0.5cm}

       Instructor: Dr. Rebenitsch

       \vspace{0.5cm}

       Section 1 DUE: Thursday, Aug 27th, at 7AM \\ 
       Section 2 DUE: Thursday, Sept 3 th, at 7AM  

       \vfill

       Department of Computer Science and Engineering\\
       South Dakota School of Mines and Technology\\

   \end{center}
\end{titlepage}
%------------------------------------------------------------------------------------
\newpage
\subsection*{Section 2: Recursion Analysis}

\textbf{1. (26 pt) Determine the run time (bit-O) for the following recurrence formula using the tree or substitution method.
You may use the master method only to check your answer. } \\ 

\[
  T(n) =  
  \begin{cases}
    1 & n = 1  \\
    3T(\frac{n}{4}) + n^2 & n > 1  \\
  \end{cases}
\]

Substitution Method: \\ 

1. We guess that the form of the solution is $ T(n) = O(n^2 log(n)) $.

2. We want to show by mathematical induction that $ T(n) \leq d \cdot n^2 log(n) $ for some constant $ d > 0 $. 

\[
  T(n) = 3T \left(\frac{n}{4} \right) + n^2
\]
\begin{center}
  Now substitute for the $T \left(\frac{n}{4} \right)$ term in the above equation:
  \end{center}
\[
  T(n) = 3 \cdot d \left(\frac{n}{4} \right)^2 \cdot log \left(\frac{n}{4} \right) + n^2
\]
\[
  = \frac{3}{16} d \cdot n^2 \cdot log \left(\frac{n}{4} \right) + n^2
\]
\[
  \leq d \cdot n^2 \cdot log(n) + n^2
\]
\begin{center}
  When $d$ is sufficiently large we drop the $ n^2 $ term:
\end{center}
\[
  d \cdot n^2 \cdot log(n)
\]
\\

\textbf{2. (26 pt) Determine the run time (big-O) for the following recurrence formula using the tree or substitution method.
You may use the master method only to check your answer.} \\

\[
  T(n) =  
  \begin{cases}
    1 & n = 1  \\
    2T(\frac{n}{3}) + n^3 & n > 1  \\
  \end{cases}
\]

Substitution Method: \\ 

1. We guess that the form of the solution is $ T(n) = O(n^3 log(n)) $.

2. We want to show by mathematical induction that $ T(n) \leq d \cdot n^3 log(n) $ for some constant $ d > 0 $. 

\[
  T(n) = 2T \left(\frac{n}{3} \right) + n^3
\]
\begin{center}
  Now substitute for the $T \left(\frac{n}{3} \right)$ term in the above equation:
\end{center}
\[
  T(n) = 2 \cdot d \left(\frac{n}{3} \right)^3 \cdot log \left(\frac{n}{3} \right) + n^3
\]
\[
  = \frac{1}{3} d \cdot n^3 \cdot log \left(\frac{n}{3} \right) + n^3
\]
\[
  \leq d \cdot n^3 \cdot log(n) + n^3
\]
\begin{center}
  When $d$ is sufficiently large we drop the $ n^3 $ term:
\end{center}
\[
  d \cdot n^3 \cdot log(n)
\]
\newpage

\textbf{3. (12 pt) Determine which case of the Master Theorem applies for the following recurrences. Include the values of
a, b, and k (and ideally $b^k$) as proof of your selection. Also, include the final big-theta formula. 
You also have teh option of a recurrence relation that cannot use the master method as described in class, 
in which case, state it "fails". }

  a. $ T(n) = 2T \left( \frac{n}{2} \right) + \sqrt{n} $ \\

  \[
    a=2, b=2, f(n)=\sqrt(n)
  \]

  b. $ T(n) = 3T \left( \frac{n}{4} \right) + n^2 $ \\

  \[
    a=3, b=4, f(n)=n^2
  \]

  c. $ T(n) = 2T \left( \frac{n}{4} \right) + T \left( \frac{n}{2} \right) + n^3 $ \\

  \[
    a=2, b=2, f(n) = n^3
  \]


  d. $ T(n) = 16T \left( \frac{n}{2} \right) + n^4 $ \\

  \[
    a=16, b=2, f(n) = n^4
  \]


\newpage
\textbf{4. (10 pt) Prove the correctness of the outer loop of the following 2D array summation using
the loop invariant technique.}

\begin{lstlisting}
  sum_array(A[][])
      sum = 0
      for each row R
          for each element j in R
              sum = sum + j
      return sum
\end{lstlisting}

\textbf{Initialization:} The variable called sum is set to 0 before any cells in the matrix were visited. \\

\textbf{Maintenance:} Let $x$ be the current row in the matrix and let $y$ be the current column in the matrix.
Let's use zero-indexing. At cell $A[x][y]$ in the matrix, we know that, for rows 0 to x-1, every element in every column
has been added to the sum. And, we know that, for the current row, x, every element from 
columns 0 to y has been added to the sum.  \\

\textbf{Termination:} When the outer loop terminates, all rows have been visited. Since every cell is visited 
when a single row is visited, we know that every cell has been visited. Therefore, sum contains the summation of
every cell in array A. \\

\textbf{5) (8 pt) For the following problem, what is the best sort among the given list. This is a real-
world problem, so some ambiguity exists. Therefore, you must also list any assumptions,
and explain why you removed the other sort options. Your options are insertion, selection,
bubble, merge, quick, and heap. Consider stability, in-place, data structure, etc. in your
answer: }\\ 

\begin{center}
  Problem: You want to sort the number of steps taken per day in the last 2 years on a smart
watch.
\end{center}

\textbf{Assumptions:} The watch is not a smart watch. The watch has little memory.  \\

Two years of data is $ n = 365 * 2 = 730 $.
So, the most important factor in choosing which algorithm to use is memory. 
Quicksort is not a practical choice because it is a recursive algorithm and would 
require a structure such as a stack to keep track of each function call. A stack is not available
on our watch.
Mergesort requires O(n log(n)) extra memory. 

Another important factor in choosing the right sorting algorithm is speed. 
So, we are left with four sorting algorithms which are all in-place.
In theory, an O(n log(n)) algorithm would sort the steps in $T(n) = 730 log(730) $ operations
while an $O(n^2)$ algorithm would sort the list of steps in $T(n) = 730 * 730 $ operations.
Therefore it would be faster to sort using a log-linear sorting algorithm. 
Consequently, we choose Heapsort because it is faster than Bubble Sort, Insertion Sort, Selection Sort.
Heapsort is an in-place sorting algorithm. On average it performs at O(n log(n)). The worst case 
performance of heapsort is O(n log(n)). 

\newpage
\textbf{6. (18 pt) Write the resulting recurrence relation (the T(n) piece-wise function) for the
following pseudocode where A is an arrays of integers: } \\

\begin{lstlisting}
  FUNC(A, s, e)
      if s >= e
          print s, e, and all of A[1..n]
          return
      cut1 = (e - s) / 2
      cut2 = (e - s) / 4
      FUNC(A, s, s + cut1)
      FUNC(A, s + cut1 + 1, s + cut1 + cut2)
      FUNC(A, s + cut1 + cut2 + 1, e)
\end{lstlisting} 

\textbf{Solution:}  

\[
  T(n) = T \left( \frac{n}{2} \right) + 2T \left( \frac{n}{4} \right) + n;
\]

\textbf{Explanation:} First, an assumption that we make in our analysis is that
array, A, is passed by reference so that A doesn't have to be copied 
for every function call. 
There are three recursive function calls in FUNC. 
cut1 is half of the distance between the start, $s$, and end, $e$, of the current segment 
of the array. cut2 is one forth of the distance between the start and end
of the current segment of the array.
Each of the three recursive function calls pass arguments which effectivly shorten the range 
between the start and end of the array.
The first function call, FUNC(A, s, s + cut1), essentially passes the first half of the current
segment of A.  
The second and third function calls, FUNC(A, s + cut1 + 1, s + cut1 + cut2) and 
FUNC(A, s + cut1 + cut2 + 1, e), both essentially pass a quarter of the current segment of 
the array.
Thus: 

\[ 
  T(n/2) + T(n/4) + T(n/4) = T(n/2) + 2T(n/4)
\]

Since the whole array is printed if $ s \geq e $ for any given function call, we add $ n $ to the total runtime 
of $T$.


\end{document}





