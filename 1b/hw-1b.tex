\documentclass[8pt, a4paper]{article}
\usepackage{listings}
\usepackage{pdfpages}
\usepackage[legalpaper, margin=1in]{geometry}
\usepackage{amsmath}

%------------------------------------------------------------------------------------
\begin{document}
\begin{titlepage}
   \begin{center}
       \vspace*{1cm}
       \large
       \textbf{Homework 1: Introduction to Algorithmic Analysis and Recurrence}
       \normalsize

       \vspace{0.5cm}

       \textbf{Author: Gabriel Hofer}

       \vspace{0.5cm}

       CSC-372 Analysis of Algorithms

       \vspace{0.5cm}

       Instructor: Dr. Rebenitsch

       \vspace{0.5cm}

       Section 1 DUE: Thursday, Aug 27th, at 7AM \\ 
       Section 2 DUE: Thursday, Sept 3 th, at 7AM  

       \vfill

       Department of Computer Science and Engineering\\
       South Dakota School of Mines and Technology\\

   \end{center}
\end{titlepage}
%------------------------------------------------------------------------------------
\newpage
\subsection*{Section 2: Recursion Analysis}

1) (26 pt) Determine the run time (bit-O) for the following recurrence formula using the tree or substitution method.
You may use the master method only to check your answer. \\ 

\[
  T(n) =  
  \begin{cases}
    1 & n = 1  \\
    2T(\frac{n}{3}) + n^3 & n > 1  \\
  \end{cases}
\]

We assume or ``guess'' that the solution is $ T(n) = O(n^3 log(n)) $.

We want to show that $ T(n) \leq d \cdot n^3 log(n) $ for some constant $ d > 0 $.





2) (26 pt) Determine the run time (big-O) for the following recurrence formula using the tree or substitution method.
You may use the master method only to check your answer. \\


3) (12 pt) Determine which case of the Master Theorem applies for the following recurrences. Include the values of
a, b, and k (and ideally $b^k$) as proof of your selection. Also, include the final big-theta formula. 
You also have teh option of a recurrence relation that cannot use the master method as described in class, 
in which case, state it "fails". \\

  a. $ T(n) = 2T (\frac{n}{2}) + \sqrt{n} $




\end{document}







